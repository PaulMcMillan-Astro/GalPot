\documentclass{article}
\usepackage{natbib,fleqn}
%\usepackage{harvard}

\def\genf{generating function }
\def\be{\begin{equation}}
\def\ee{\end{equation}}

\setlength{\textwidth}{15.5cm}
\setlength{\textheight}{23cm}
\setlength{\oddsidemargin}{0.3cm}
\setlength{\topmargin}{-1cm}

\begin{document}

\begin{center}
{\huge GalPot (Galaxy potential following falcON)}
\end{center}


This documentation is just here to explain how to compile and give parameters to
GalPot. See Section 2.3 of \citet{DB98} for a fuller explanation of how it works.
See testGalPot for examples of what the code can do.

\section{Compilation}
``make" makes the library and example executables.

\noindent``make MyExecutable.exe" will compile a new executable written by the user and called ``MyExecutable.cc" 

\noindent The command ``make GalPot'' just makes the GalPot libraries. 

 {\bf N.B.} In all cases here, the length unit is kpc, the time unit is Myr, and the mass unit is
M$_\odot$. For reference, $G=4.49866\times
10^{-12}$kpc$^3$Myr$^{-2}$M$_\odot^{-1}$. A velocity of 1 kpc Myr$^{-1}
  = 977.77$km s$^{-1}$ (See Units.h for further).
  
\section{Input} 
The file that is given as the input file for the potential must be of
the following form (without the comment lines):
\vspace{5mm}

\begin{tabular}{lllllll}
3      &     &      &     &    &     &\# No. Disks ($0\leq{\rm No.disks}\leq3$) \\
8.90e7 & 1.8 & 0.04 & 0   & 0  &     &\# Param. for each of the three disks.\\
3.50e7 & 1.2 & -0.15& 0   & 0  &     &\# {\bf N.B} {\it essential} to
give 5 param.\\  
1.30e7 & 1.1 & 0.25 & 0   & 0  &     &\# No. lines of param. = No. disks\\
2      &     &      &     &    &     &\# No. Spheroids ($0\leq{\rm
  No. sph.}\leq2$) \\
4.0e7   & 0.8 & -0.5 & 1.4 & 10 & 50 &\#  Param. for each spheroid. \\ 
2.0e7   & 0.2 & -1.5 & 3.4 & 10 & 50 &\# {\it essential} to give 6 parameters\\
\end{tabular}


\subsection{Disk}

The disk parameters, as given in that file are (in order) $\Sigma_0$,
$R_d$, $z_d$, $R_0$, $\epsilon$. 

$\Sigma_0$ is the disk's central surface density (in M$_\odot$kpc$^{-2}$
and the absence of a cutoff), $R_d$ is the disk
scale radius, $z_d$ its scale height (though note there is a
difference between negative and positive values), and $R_0$ is an inner
cutoff radius (all in kpc). The term $\epsilon$ perturbs the disc from a pure exponential, with a peak fractional change in surface density of $\sim\epsilon$ and a scale length of $2R_d$. In numerical terms (cylindrical polars), these
give a surface density
\begin{equation}
\Sigma(R)=\Sigma_0\;\textrm{exp}\left(-\frac{R_0}{R}-\frac{R}{R_d}+
  \epsilon\textrm{cos}\left(\frac{\pi R}{R_d}\right)\right),
\end{equation}
i.e. a standard exponential disk with optional hole in the middle and/or
$\epsilon$ term modulation.

The vertical structure of the density is either an exponential in
$|z|$ (if $z_d>0$), or isothermal (if $z_d<0$), i.e.
\begin{equation}
\rho_d(R,z)=\left\{\begin{array}{lc}\frac{\Sigma(R)}{2z_d}\,\textrm{exp}\left(\frac{-\mid
        z\mid}{z_d}\right) & \textrm{for }z_d > 0 \\
\frac{\Sigma(R)}{4(-z_d)}\,\textrm{sech}^2\left(\frac{z}{2\,z_d}\right)
& \textrm{for } z_d
< 0.\\\end{array}\right. 
\end{equation}
{\bf N.B.} There is a factor of two in the denominator of the sech$^2$
profile. That is so that they tend to the same thing as
$z\rightarrow\infty$.

\subsection{Spheroids}

The spheroid parameters are (in order) $\rho_0$, $q$, $\gamma$,
$\beta$, $r_0$, $r_{cut}$.

$\rho_0$ is a scale density (in M$_\odot$kpc$^{-3}$; $q$ is the axis ratio ($q<1$ is flatter
than a sphere, $q>1$ is prolate); $\gamma$ is the inner density slope,
$\beta$ the outer density slope; $r_0$ is a scale radius and $r_{cut}$
is a cutoff radius (both in kpc). This corresponds to a density profile
\begin{equation}
\rho_s=\frac{\rho_0}{(r^\prime/r_0)^\gamma(1+r^\prime/r_0)^{\beta-\gamma}}\; 
\textrm{exp}\left[-\left(r^\prime/r_{cut}\right)^2\right],
\end{equation}
where, in cylindrical coordinates
\begin{equation}
r^\prime = \sqrt{R^2 + (z/q)^2}
\end{equation}

\section{Use}
The example executable testGalPot gives examples of the use of the GalaxyPotential class.


\begin{thebibliography}{36}
 

\bibitem[\protect\citeauthoryear{Dehnen \& Binney}{1998}]{DB98}
{Dehnen} W., {Binney} J., 1998, MNRAS, 294, 429
 
\end{thebibliography}
 
\end{document}

